\documentclass{article}
\usepackage[utf8]{inputenc}
\usepackage{graphicx}
\usepackage{tabularx}
\usepackage{ctex}
\usepackage{booktabs} 
\usepackage{enumerate}
\usepackage{hyperref}


\title{电脑远程连接}
\author{小李飞刀}
\date{\today}

\begin{document}

\maketitle

\section{背景}
由于游戏本太过于笨重,而又不想买工作本的大学牲来说,用一个平板和一种直接与
电脑相连的应用软件就可以解决大部分问题
\\ \indent 因此,这个软件我选择了ToDesk
\section{下载}
首先,ToDesk存在两种版本,免费版和专业版,那我们就分析一下两款功能上有什么差异
\\  免费版
\begin{enumerate}[(1)]
    \item 基础功能‌:ToDesk免费版提供了基础的远程桌面连接功能,支持跨设备跨系统连接,具备隐私屏等功能。‌
    \item 多平台兼容‌:免费版支持Windows、macOS、Linux以及移动设备。‌
    \item 用户每月最多可以发起300次连接,且每月连接时长最多为120小时
\end{enumerate}
 付费版
\begin{enumerate}[(1)]
    \item 高级功能‌:专业版在免费版的基础上增加了多会话管理、高级权限控制、远程打印、会话录制与回放、文件同步与备份等功能。这些功能特别适合企业用户,能够提升工作效率和保障数据安全。
    \item 多设备支持‌:专业版允许用户同时控制多台设备,极大提升了工作效率。
    \item 用户可以享受不限时效的远程服务‌
\end{enumerate}
作为学生党来说,我认为免费版就已经足够日常使用了,如果需要更久的时长,以及连接更多的电脑端,那我认为选择付费版是比较合适的
\section{下载中}
下面我们讲讲如何下载
\\ \indent 我们需要在平板与电脑上都下载
\\ \indent 电脑端
\\ \indent ToDesk官网网址\href{https://www.todesk.com}{https://www.todesk.com}

\begin{figure}[h]
    \centering
    \includegraphics[width=0.5\textwidth]{fi/屏幕截图 2025-03-14 220847.jpg}
\end{figure} 
 \indent 点击个人版免费下载
 \begin{figure}[h]
    \centering
    \includegraphics[width=0.5\textwidth]{fi/屏幕截图 2025-03-14 222136.jpg}
\end{figure} 
\indent 点击打开文件
\begin{figure}[h]
    \centering
    \includegraphics[width=0.5\textwidth]{fi/屏幕截图 2025-03-14 222233.jpg}
\end{figure}
\indent 可以快速安装,安装在C盘,也可以进行自定义,将软件安转在自己喜欢的位置
\begin{figure}[h]
    \centering
    \includegraphics[width=0.5\textwidth]{fi/屏幕截图 2025-03-14 222250.jpg}
\end{figure}
\indent 我安装在了D盘
\\手机端
\indent ToDesk是一种国产应用软件,因此可以在各大应用商城中找到,如果找不到的话也可以在官网中下载
\section{连接}
现在我们已经下载好了,接下来进行连接
\begin{figure}[h]
    \centering
    \includegraphics[width=0.5\textwidth]{fi/屏幕截图 2025-03-14 222438.jpg}
\end{figure}
\indent 电脑端打开后是这种画面,将设备代码输入平板的连接中,然后是密码
\begin{figure}[h]
    \centering
    \includegraphics[width=0.5\textwidth]{fi/屏幕截图 2025-03-14 222528.jpg}
\end{figure}
\indent 密码的话,临时代码会定时变化,我这里选择自己设置安全密码,按照图片上设置即可,接下来进行连接就行
\section{完毕}
现在大家就可以用平板远程进行连接电脑进行学习了
\end{document} 