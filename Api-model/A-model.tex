\documentclass{article}
\usepackage[utf8]{inputenc}
\usepackage{graphicx}
\usepackage{tabularx}
\usepackage{ctex}
\usepackage{booktabs} 
\usepackage{enumerate}
\usepackage{hyperref}
\usepackage{listings}
\usepackage[dvipsnames]{xcolor}
\lstset{
    language=Python, % 设置语言
 basicstyle=\ttfamily, % 设置字体族
 breaklines=true, % 自动换行
 keywordstyle=\bfseries\color{NavyBlue}, % 设置关键字为粗体,颜色为 NavyBlue
 morekeywords={}, % 设置更多的关键字,用逗号分隔
 emph={self}, % 指定强调词,如果有多个,用逗号隔开
    emphstyle=\bfseries\color{Rhodamine}, % 强调词样式设置
    commentstyle=\itshape\color{black!50!white}, % 设置注释样式,斜体,浅灰色
    stringstyle=\bfseries\color{PineGreen!90!black}, % 设置字符串样式
    columns=flexible,
    numbers=left, % 显示行号在左边
    numbersep=2em, % 设置行号的具体位置
    numberstyle=\footnotesize, % 缩小行号
    frame=single, % 边框
    framesep=1em % 设置代码与边框的距离
}

\title{抢渡长江的数学建模分析}
\author{李俊霖}
\date{\today}

\begin{document}

\maketitle

\section{任务}
我将用python构建一个基于llama 2的聊天机器人,然后用Api对于它进行调用
\section{结构}
用 Streamlit 作为前端,用 Python 构建一个 Llama 2 聊天机器人,而 LLM 后端则通过对 Replicate 上托管的 Llama 2 模型的 API 调用来处理
\section{获取Replicate API 令牌}
\begin{enumerate}[(1)]
    \item 转到Replicate\href {https://llama2.streamlit.app}{ https://llama2.streamlit.app}
    \item 使用您的 GitHub 帐户登录。
    \item 转到 API 令牌页面并复制您的 API 令牌。
\end{enumerate}
\begin{figure}[h]
    \centering
    \includegraphics[width=0.5\textwidth]{fi/屏幕截图 2025-03-14 231037.jpg}
\end{figure}
新建一个Api-Token,这个Api-Token就是之后你进行远程调用时所需要的 
\section{设置编码环境}
本地开发
\\ \indent 要设置本地编码环境,请在命令行提示符中输入以下命令:

\begin{lstlisting}
    pip install streamlit replicate
\end{lstlisting}

\section{构建应用程序} 
总代码
\begin{lstlisting}
import streamlit as st
import replicate
import os

# App title
st.set_page_config(page_title="🦙💬 Llama 2 Chatbot")

# Replicate Credentials
with st.sidebar:
    st.title('🦙💬 Llama 2 Chatbot')
    if 'REPLICATE_API_TOKEN' in st.secrets:
        st.success('API key already provided!', icon='✅')
        replicate_api = st.secrets['REPLICATE_API_TOKEN']
    else:
        replicate_api = st.text_input('Enter Replicate API token:', type='password')
        if not (replicate_api.startswith('r8_') and len(replicate_api)==40):
            st.warning('Please enter your credentials!', icon='⚠️')
        else:
            st.success('Proceed to entering your prompt message!', icon='👉')
    os.environ['REPLICATE_API_TOKEN'] = replicate_api

    st.subheader('Models and parameters')
    selected_model = st.sidebar.selectbox('Choose a Llama2 model', ['Llama2-7B', 'Llama2-13B'], key='selected_model')
    if selected_model == 'Llama2-7B':
        llm = 'a16z-infra/llama7b-v2-chat:4f0a4744c7295c024a1de15e1a63c880d3da035fa1f49bfd344fe076074c8eea'
    elif selected_model == 'Llama2-13B':
        llm = 'a16z-infra/llama13b-v2-chat:df7690f1994d94e96ad9d568eac121aecf50684a0b0963b25a41cc40061269e5'
    temperature = st.sidebar.slider('temperature', min_value=0.01, max_value=5.0, value=0.1, step=0.01)
    top_p = st.sidebar.slider('top_p', min_value=0.01, max_value=1.0, value=0.9, step=0.01)
    max_length = st.sidebar.slider('max_length', min_value=32, max_value=128, value=120, step=8)
    st.markdown('📖 Learn how to build this app in this [blog](https://blog.streamlit.io/how-to-build-a-llama-2-chatbot/)!')

# Store LLM generated responses
if "messages" not in st.session_state.keys():
    st.session_state.messages = [{"role": "assistant", "content": "How may I assist you today?"}]

# Display or clear chat messages
for message in st.session_state.messages:
    with st.chat_message(message["role"]):
        st.write(message["content"])

def clear_chat_history():
    st.session_state.messages = [{"role": "assistant", "content": "How may I assist you today?"}]
st.sidebar.button('Clear Chat History', on_click=clear_chat_history)

# Function for generating LLaMA2 response. Refactored from https://github.com/a16z-infra/llama2-chatbot
def generate_llama2_response(prompt_input):
    string_dialogue = "You are a helpful assistant. You do not respond as 'User' or pretend to be 'User'. You only respond once as 'Assistant'."
    for dict_message in st.session_state.messages:
        if dict_message["role"] == "user":
            string_dialogue += "User: " + dict_message["content"] + "\n\n"
        else:
            string_dialogue += "Assistant: " + dict_message["content"] + "\n\n"
    output = replicate.run('a16z-infra/llama13b-v2-chat:df7690f1994d94e96ad9d568eac121aecf50684a0b0963b25a41cc40061269e5', 
                           input={"prompt": f"{string_dialogue} {prompt_input} Assistant: ",
                                  "temperature":temperature, "top_p":top_p, "max_length":max_length, "repetition_penalty":1})
    return output

# User-provided prompt
if prompt := st.chat_input(disabled=not replicate_api):
    st.session_state.messages.append({"role": "user", "content": prompt})
    with st.chat_message("user"):
        st.write(prompt)

# Generate a new response if last message is not from assistant
if st.session_state.messages[-1]["role"] != "assistant":
    with st.chat_message("assistant"):
        with st.spinner("Thinking..."):
            response = generate_llama2_response(prompt)
            placeholder = st.empty()
            full_response = ''
            for item in response:
                full_response += item
                placeholder.markdown(full_response)
            placeholder.markdown(full_response)
    message = {"role": "assistant", "content": full_response}
    st.session_state.messages.append(message)
\end{lstlisting} 
\section{我部署时解决的问题}
\begin{enumerate}[(1)]% 次级序号
    \item 首先,创建一个文件夹,用vscode打开文件
    \item 设置.env虚拟环境
    \item 然后按照部署相关环境,比如streamlit和republic
    \item 之后在此文件夹中新建一个文件夹.streamlit
    \item 在.streamlit文件夹中新建secrets.toml文件,然后如照片那样在其中输入之前的Api
    \begin{figure}[h]
        \centering
        \includegraphics[width=0.5\textwidth]{fi/微信图片_20250314233640.jpg}
    \end{figure}
    \item 最后进行调用,记住一定要用
    \begin{lstlisting}
    streamlit run main.py
    \end{lstlisting} 
    最后成功了
    \end{enumerate}
\section{声明}
此篇文章深度借鉴了{https://blog.streamlit.io/how-to-build-a-llama-2-chatbot/}{如何构建 Llama 2 聊天机器人}这篇文章,这篇文章只是记录一下自己的学习过程与学习心得,希望对于有所帮助
\end{document}