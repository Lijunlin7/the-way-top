\documentclass{article}
\usepackage[utf8]{inputenc}
\usepackage{graphicx}
\usepackage{tabularx}
\usepackage{ctex}
\usepackage{booktabs} 
\usepackage{enumerate}
\usepackage{hyperref}
\usepackage{listings}
\usepackage[dvipsnames]{xcolor}
\lstset{
    language=Python, % 设置语言
 basicstyle=\ttfamily, % 设置字体族
 breaklines=true, % 自动换行
 keywordstyle=\bfseries\color{NavyBlue}, % 设置关键字为粗体,颜色为 NavyBlue
 morekeywords={}, % 设置更多的关键字,用逗号分隔
 emph={self}, % 指定强调词,如果有多个,用逗号隔开
    emphstyle=\bfseries\color{Rhodamine}, % 强调词样式设置
    commentstyle=\itshape\color{black!50!white}, % 设置注释样式,斜体,浅灰色
    stringstyle=\bfseries\color{PineGreen!90!black}, % 设置字符串样式
    columns=flexible,
    numbers=left, % 显示行号在左边
    numbersep=2em, % 设置行号的具体位置
    numberstyle=\footnotesize, % 缩小行号
    frame=single, % 边框
    framesep=1em % 设置代码与边框的距离
}

\title{wsl虚拟机部署流程}
\author{李俊霖}
\date{\today}

\begin{document}

\maketitle

\section{任务}
我将基于Windows系统部署wsl并下载虚拟机ubuntu,搭建linux虚拟环境
\section{wsl优势}
通过询问deepseek,我们能更加全面的获知wsl相比于虚拟机的优势
\\1. 性能更高
\\ \indent  WSL:直接与 Windows 内核集成,无需虚拟化硬件,因此性能接近原生 Linux。
\\ \indent 虚拟机:需要虚拟化整个操作系统和硬件,性能开销较大,尤其是在 CPU 和 I/O 密集型任务中。
\\2. 资源占用更少
\\ \indent WSL:轻量级,仅运行 Linux 用户空间,内存和 CPU 占用较低。
\\ \indent 虚拟机:需要为虚拟化环境分配独立的资源(如内存、CPU 和存储),资源占用较高。
\\3. 启动速度更快
\\ \indent WSL:启动几乎是瞬时的,因为不需要启动完整的操作系统。
\\ \indent 虚拟机:需要启动完整的 Linux 内核和操作系统,启动时间较长。
\\4. 与 Windows 系统无缝集成
\\ \indent WSL:
\\ \indent 可以直接访问 Windows 文件系统(如 /mnt/c 对应 Windows 的 C 盘)。
\\ \indent 支持在 Windows 和 Linux 之间无缝调用命令行工具。
\\ \indent 支持在 Windows 终端中运行 Linux 命令。
\\ \indent 虚拟机:文件系统和网络通常是隔离的,需要通过共享文件夹或网络配置来实现交互。
\\5. 开发体验更佳
\\ \indent WSL:
\\ \indent 支持直接运行 Linux 开发工具(如 GCC、Python、Node.js 等)。
\\ \indent 支持与 Visual Studio Code 等开发工具深度集成。
\\ \indent 可以在 Windows 上直接运行 Linux GUI 应用(WSLg)。
\\ \indent 虚拟机:需要额外的配置才能实现类似的功能,且体验不如 WSL 流畅。
\\6. 无需管理虚拟化软件
\\ \indent WSL:直接集成在 Windows 中,无需安装和管理额外的虚拟化软件(如 VMware、VirtualBox)。
\\ \indent 虚拟机:需要安装和配置虚拟化软件,增加了复杂性。
\\7. 更适合轻量级 Linux 任务
\\ \indent WSL:适合运行命令行工具、脚本、开发环境等轻量级任务。
\\ \indent 虚拟机:更适合需要完整 Linux 桌面环境或复杂系统配置的场景。
\\8. 支持 GPU 加速(WSL 2)
\\ \indent WSL 2:支持 GPU 加速,适合机器学习、数据科学等需要 GPU 计算的任务。
\\ \indent 虚拟机:GPU 虚拟化支持有限,配置复杂且性能较差。
\\9. 更低的系统开销
\\ \indent WSL:由于不需要虚拟化硬件,系统开销更低,适合低配置设备。
\\ \indent 虚拟机:需要为虚拟化环境分配独立的资源,对系统配置要求较高。
\\10. 易于安装和配置
\\ \indent WSL:通过 Microsoft Store 或命令行即可快速安装,配置简单。
\\ \indent 虚拟机:需要下载 ISO 镜像、安装操作系统、配置虚拟硬件等,步骤繁琐。

\section{部署步骤}
\subsection{安装WSL}
确认虚拟化已启用
\\ \indent 检查任务管理器
\begin{enumerate}[(1)]% 次级序号
    \item 打开任务管理器(Ctrl + Shift + Esc 或右键点击任务栏选择“任务管理器”)。
    \item 转到 性能(Performance)选项卡。
    \item 在 CPU 栏目下方,应看到 虚拟化(Virtualization)显示为 启用(Enabled)。
    \begin{figure}[h]
        \centering
        \includegraphics[width=0.5\textwidth]{fi/屏幕截图 2025-03-16 191706.jpg}
    \end{figure}
\end{enumerate}
如果没有启用,进行以下步骤
\begin{enumerate}[(1)]% 次级序号
    \item 打开 PowerShell 以管理员身份运行。
    
    \item 启用 WSL 功能:
    \begin{lstlisting}
        dism.exe /online /enable-feature /featurename:Microsoft-Windows-Subsystem-Linux /all /norestart
    \end{lstlisting}

    \item  3. 启用虚拟机平台(WSL 2 所必需):
    \begin{lstlisting}
        dism.exe /online /enable-feature /featurename:VirtualMachinePlatform /all /norestart
    \end{lstlisting}
\end{enumerate}
 \indent 下载WSL2
\begin{lstlisting}
    wsl --set-default-version 2
\end{lstlisting}

\section{安装Ubuntu}
在微软软件商店进行下载下载如图的版本(命令行界面)
\begin{figure}[h]
    \centering
    \includegraphics[width=0.5\textwidth]{fi/屏幕截图 2025-03-16 192615.jpg}
\end{figure}
\\ \indent 下载后打开,会进行用户名的设置和密码的设置,注意:用户名不能大写,密码会被隐藏,但会进行二次确认
\begin{figure}[h]
    \centering
    \includegraphics[width=0.5\textwidth]{fi/屏幕截图 2025-03-16 192632.jpg}
\end{figure}
\\ \indent 点击如图的倒三角形状,然后点击Ubuntu,就会打开linux环境。
然后就一切好了
\section{声明}
此篇文章深度借鉴了{https://blog.csdn.net/yueyuanhuaqing/article/details/140377645}{WSL 安装}这篇文章,这篇文章只是记录一下自己的学习过程与学习心得,希望对于有所帮助
\end{document}