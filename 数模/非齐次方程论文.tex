\documentclass{article}
\usepackage[utf8]{inputenc}
\usepackage{graphicx}
\usepackage{tabularx}
\usepackage{ctex}
\usepackage{gensymb}

\title{非齐次方程练习}
\author{李俊霖}
\date{\today}

\begin{document}

\maketitle
\section{问题提出}
已知一颗炮弹做斜上抛运动,初速度大小为200m/s,击中水平距离360m、垂直距离160m的目标,问当忽略空气阻力时,炮弹发射角多大?
请给出这道高中物理题的答案,并画出炮弹飞行的轨迹。
\\ \indent 要求: 提交一份word或pdf文档,需包含模型假设、模型建立、模型求解(需包含发射角的大小和炮弹飞行轨迹图)和附录这几部分。附录中要放上你用来求解模型的代码。  
\section{模型假设}
1. 忽略空气阻力:炮弹在飞行过程中只受重力作用
\\ \indent 2. 重力加速度恒定:重力加速度 g=9.81m/s
\\ \indent 3. 初始速度已知:炮弹的初速度大小为 v0=200m/s
\\ \indent 4. 目标位置已知:炮弹需要击中水平距离 x=360m,垂直距离 y=160m的目标。
\section{模型建立}
炮弹的运动可以分解为水平方向和垂直方向的运动:
\\ \indent 水平方向:初速度分量:
\begin{eqnarray}
    V_{0x}=v_0*\cos(a)
\end{eqnarray}
运动方程:
\begin{eqnarray}
    x=V_{0x}*t
\end{eqnarray}
\\ \indent 垂直方向:
初速度分量:
\begin{eqnarray}
    V_{0y}=v_0*\sin(a)
\end{eqnarray}
运动方程:
\begin{eqnarray}
    y=V_{0y}*t-0.5*g*t^2
\end{eqnarray}
\\ \indent 其中,a为发射角,t 为飞行时间

\section{模型求解}

\subsection{求解流程}
我们需要求解发射角θ 和飞行时间 t
\\ \indent 水平方向方程:
\begin{eqnarray}
    x=v_0*\cos(a)*t
\end{eqnarray}
解得:
\begin{eqnarray}
   t=\frac{x}{v_0*\cos(a)}
\end{eqnarray}
\\ \indent 垂直方向方程:
\begin{eqnarray}
    y=v_0*\sin(a)*t-0.5*g*t^2
\end{eqnarray}
将 t 代入:
\begin{eqnarray}
    y=v_0*\sin(a)*\frac{x}{v_0*\cos(a)}-0.5*g*(\frac{x}{v_0*\cos(a)})^2
\end{eqnarray}
化简得:
\begin{eqnarray}
    y=x*\tan(a)-\frac{gx^2}{2v_0^2*(cos(a))^2}
\end{eqnarray}
利用三角恒等式 
\begin{eqnarray}
    \frac{1}{cos(a)^2}=1+\tan(a)^2
\end{eqnarray}
代入得:
\begin{eqnarray}
    y=x*\tan(a)-\frac{gx^2}{2*v_0^2}*(1+\tan(a)^2)
\end{eqnarray}
这是一个关于 tanθ 的二次方程:
代入已知数值:x=360,y=160
计算得:
\begin{eqnarray}
    31.86*\tan(a)^2-360*\tan(a)+191.86=0
\end{eqnarray}
观察得f(0)>0且f(1)<0,f(10)<0且f(11)>0
运用二分法逼近得到在(0,1),(10,12)这两个区间中分别有一根

\subsection{MATLAB代码如图}
\begin{figure}[h]
    \centering
    \includegraphics[width=0.5\textwidth]{fi/屏幕截图 2025-03-08 181209.jpg}
\end{figure}
\subsection {结果如图}
\begin{figure}[h]% 插入两张图片并且并排
	\centering
	\begin{minipage}{0.48\textwidth}
		\centering
		\includegraphics[width=0.83\textwidth]{fi/屏幕截图 2025-03-08 180728.jpg}
	\end{minipage}
	\hspace{0cm}% 图片间距
	\hfill% 撑满整行
	\begin{minipage}{0.48\textwidth}
		\centering
		\includegraphics[width=0.83\textwidth]{fi/屏幕截图 2025-03-08 180745.jpg}
	\end{minipage}
\end{figure}
\subsection{结果验证} 

\begin{figure}[h]% 插入两张图片并且并排
	\centering
	\begin{minipage}{0.48\textwidth}
		\centering
		\includegraphics[width=0.83\textwidth]{fi/屏幕截图 2025-03-08 180942.jpg}
		\caption{\fontsize{10pt}{15pt}\selectfont 图一 X1}
	\end{minipage}
	\hspace{0cm}% 图片间距
	\hfill% 撑满整行
	\begin{minipage}{0.48\textwidth}
		\centering
		\includegraphics[width=0.83\textwidth]{fi/屏幕截图 2025-03-08 181132.jpg}
		\caption{\fontsize{10pt}{15pt}\selectfont 图二 X2}
	\end{minipage}
\end{figure}
 可知:tan(a)=11.055或tan(a)=0.55则发射角为 a=84.8\textdegree 或a=28.8\textdegree
\\
\subsection{将结果可视化}
\begin{figure}[h]
    \centering
    \includegraphics[width=0.5\textwidth]{fi/屏幕截图 2025-03-08 180433.jpg}
\end{figure}


\section{f附录}
\begin{figure}[h]
    \centering
    \includegraphics[width=0.5\textwidth]{fi/屏幕截图 2025-03-08 181209.jpg}
\end{figure}


\end{document}  

