\documentclass{article}
\usepackage[utf8]{inputenc}
\usepackage{graphicx}
\usepackage{tabularx}
\usepackage{ctex}

\title{抢渡长江的数学建模分析}
\author{李俊霖}
\date{\today}

\begin{document}

\maketitle

\begin{abstract}
本文是关于渡河的速度与方向预测问题,根据给出的水流速度与相关最优要求,我们可以将其视为
解决合运动的有关运动时间、位移的相关问题,
本文引入了矢量化与正交分解的研究方法,运用图形的相关特点,将复杂的运动过程分解简单运动的矢量和,
因此要达到题目局部最优的要求,就必须控制好时间与空间的一致性。
\\ \indent 针对问题一:本文构建速度的矢量三角形,第○1 小问,将已知的速度代入矢量三
角形中,综合运用正、余弦定理,就可以求出答案。
第○2 小问,以水速矢量为圆心,人速为半径画圆,与最短路径相交点即为所求。
\\ \indent 针对问题二:第○1 小问:游泳者始终以和岸边垂直的方向游,假设他能到
达终点那么他的渡江时间则由两岸的距离决定,据此可以计算其渡江速度。
第○2 小问中,为分析两次挑战赛到达终点人数百分比的差别,
本文分别求出两次挑战赛成功到达终点速度最小
值及方向,然后和以人类极限速度渡江对应的角度进行相关的比较。
\\ \indent 针对问题三:因为游泳者的速度大小不变,而水速是分段的,要想渡江成功
到达终点,本文在每个阶段将合速度进行正交分解 ,分别研究与江岸平行和垂直
的分运动,根据距离关系列出表达式,通过控制时间最短,借助 LINGO 软件的使
用求出最优解。
\\ \indent 最后,我们又随机给出一些数据,将其代入模型中,以检验模型的正确性。
\\
\\ \noindent{\textbf{关键词:}合运动;图形化;矢量化;正交分解}
\end{abstract}

\section{问题提出}
“渡江”是武汉城市的一张名片。1934 年 9 月 9 日,武汉警备旅官兵与体育
界人士联手,在武汉第一次举办横渡长江游泳竞赛活动,起点为武昌汉阳门码头,终点设在汉口三北码头,全程约 5000 米。有 44 人参加横渡,40 人达到终点,张
学良将军特意向冠军获得者赠送了一块银盾,上书“力挽狂澜”。
\\ \indent 2001 年,“武汉抢渡长江挑战赛”重现江城。2002 年,正式命名为“武汉国际
抢渡长江挑战赛”,于每年的 5 月 1 日进行。由于水情、水性的不可预测性,这种
竞赛更富有挑战性和观赏性。
\\ \indent 2002 年 5 月 1 日,抢渡的起点设在武昌汉阳门码头,终点设在汉阳南岸咀,
江面宽约 1160 米。据报载,当日的平均水温 16.8℃, 江水的平均流速为 1.89 米/
秒。参赛的国内外选手共 186 人(其中专业人员将近一半),仅 34 人到达终点,
第一名的成绩为 14 分 8 秒。除了气象条件外,大部分选手由于路线选择错误,被
滚滚的江水冲到下游,而未能准确到达终点。
\\ \indent 假设在竞渡区域两岸为平行直线, 它们之间的垂直距离为 1160 米, 从武昌汉
阳门的正对岸到汉阳南岸咀的距离为 1000 米,见示意图。
\begin{figure}[h]
    \centering
    \includegraphics[width=0.5\textwidth]{fi/屏幕截图 2025-03-01 174323.jpg}
    \caption{抢渡长江示意图}
    \label{fig:framework_comparison}
\end{figure}
\\ \indent 请你们通过数学建模来分析上述
情况, 并回答以下问题:
\\ \indent 1.○1 假定在竞渡过程中游泳者的速度
大小和方向不变,且竞渡区域每点
的流速均为 1.89 米/秒。试说明
2002年第一名是沿着怎样的路线前
进的,求她游泳速度的大小和方向。
○2 如何根据游泳者自己的速度选择
游泳方向,试为一个速度能保持在
1.5 米/秒的人选择游泳方向,并估计他的成绩。
\\ \indent 2.○1 在(1)的假设下,如果游泳者始终以和岸边垂直的方向游, 他(她)们能否到
达终点?○2 根据你们的数学模型说明为什么 1934 年 和 2002 年能游到终点的
人数的百分比有如此大的差别;○3 给出能够成功到达终点的选手的条件。
\\ \indent 3. 若流速沿离岸边距离的分布为 (设从武昌汉阳门垂直向上为 y 轴正向) :
\begin{figure}[h]
    \centering
    \includegraphics[width=0.5\textwidth]{fi/屏幕截图 2025-03-01 175348.jpg}
\end{figure} 
游泳者的速度大小(1.5 米/秒)仍全程保持不变,试为他选择游泳方向和路线,
估计他的成绩。
\\ \indent 4. 若流速沿离岸边距离为连续分布, 例如下 或你们认为合适的连续分布,如何处理这个问题。
\begin{figure}[h]
    \centering
    \includegraphics[width=0.5\textwidth]{fi/屏幕截图 2025-03-01 175403.jpg}
\end{figure} 

\section{问题分析}
在渡江问题中,我们先建立现实问题的数学模型。
\\ \indent 在问题一中,游泳者与江水的速度已给出,且为了使其能最快到达
终点,我们只需保证游泳者游泳方向沿起点与终点的连线沿直线前进即可。由此,我们将给出数据
与方向建立矢量三角形,接着代入相关数据可解出。
\\ \indent 在问题二中,若游泳者始终沿垂直岸边方向游
则速度必须达到2.19米/秒,这个速度一般人是无法达到的,因此,1934年和
2002年百分比有如此大差距的原因是游泳路线的不同和水速的差异导致的,由此建立不同
的三角形模型,代入数据求出角度
\\ \indent 在问题三中,流速离岸边距离分为三段不同分布区域,而游泳者速度不变,为抵达终点,就
必须根据水速来改变速度方向,即问题一二中的数学模型,在认定人速在不同水速区域的速度方向一定,并
并选择一组游泳者所用时间最少。由此建立数学模型。
\section{问题假设}
1、2002 年的第一名是沿最佳路线前进的。
\\ \indent 2、运动员只比速度,不限定具体的姿势,每个运动员有足够的体能到达终点。运动员在某一固定区间段方向不变。
\\ \indent 3、1934 年路线全程 5000 米视为比赛的起点到终点的距离。在 1943—2002 年间,人的游泳极限速度没有太大变化,即可用 2002 年的
人游泳极限速度来近似等于 1934 年的速度。
\\ \indent 4、在 1943 年与 2002 年,长江水速没有太大变化,即可用 2002 年的水速来近似等于 1934 年的水速。
\section{符号说明}
\begin{tabular}{|c|c|}
    \hline
    V1   &  人的游泳速度 \\
    \hline 
    V2 &合速度  \\
    \hline
    V3 &水速  \\
    \hline  
    a, b, c &三角形参数                       \\
    \hline
    T &渡江时间 \\
    \hline
    c1 &在前 200m 挑战者的游泳方向\\
    \hline
    c2 &在中间 760m 挑战者的游泳方向\\
    \hline
    c3 &在后 200m 挑战者的游泳方向  \\
    \hline
\end{tabular}

\section{建模与计算}
\subsection{问题一的建模与计算}

问题一第○1 小问:因为在竞渡过程中游泳者的速度大小和方向不变,且竞
渡区域每点的流速相同为 1.89 米/秒,且已假定 2002 年的第一名是以最佳路线前
进的,又已知第一名的渡江时间为 14 分 8 秒(884s),那么可以构建人速、水速、
合速度的矢量三角形,运用正弦定理进行求解。为此,模型如图一所示:
\begin{figure}[h]
    \centering
    \includegraphics[width=0.5\textwidth]{fi/屏幕截图 2025-03-01 212123.jpg}
\end{figure}
\\ \indent 第○1 小问的模型求解:
\begin{eqnarray}
    v2=\frac{\sqrt{1000^2+1160^2}}{848}=1.8061 
   \\ v1=\sqrt{1.89^2+1.81^2-2*1.89^2*1.81*\frac{1000}{1531.5}}=1.54
    \\a=\arcsin(\frac{1160}{1531.5})=49.24^{\circ}
    \\b=\arcsin(\sin(a)*\frac{1.89}{1.81})=68.23^{\circ}
    \\c=a+b=117.47^{\circ}
\end{eqnarray}
 即第一名的渡江速度为1,54m/s,方向 与水流方向成117.47
\\ \indent 第2小问:在速度矢量三角形中,运用正弦定理进行求解,模型如图二所示
\begin{figure}[h]
    \centering
    \includegraphics[width=0.5\textwidth]{fi/屏幕截图 2025-03-01 212123.jpg}
\end{figure}
 第2 小问的模型求解
\begin{eqnarray}
    b=\arcsin(\sin(a)*\frac{1.89}{1.5})=72.63^{\circ}
    \\c=a+b=121.87^{\circ}
    \\v2=v1*\frac{\sin(180-a-b)}{\sin(a)}=1.68
    \\t= \frac{\sqrt{1000^2+1160^2}}{v2}=15分11秒
\end{eqnarray}
即游泳者的渡江方向为:与水流方向成121.87。 ,渡江时间为 15 分 11 秒。


\subsection{问题二的建模与计算}
㈠、问题二第○1 小问:游泳者始终以和岸边垂直的方向游,假设他(她)能到达终点那么他(她)的渡江时间则由长江水速及武昌汉阳门的正对岸到汉阳南岸咀的距离决定,据此可以计算其渡江速度。模型如图三所示:
\begin{figure}[h]
    \centering
    \includegraphics[width=0.5\textwidth]{fi/屏幕截图 2025-03-01 212255.jpg}
\end{figure}
第○1 小问的模型求解:
\begin{eqnarray}
    t= \frac{1000}{v3}=529.1s
    \\t= \frac{1160}{t}=2.19
\end{eqnarray}
即当此人以 2.19m/s 的速度沿与岸边垂直的方向能游到终点,需要的时间是
529.1s,根据 2002 年游泳年终排名可知世界顶级游泳运动员 1500m 最好成绩男女平均速度为 1.6221m/s,所以在抢渡长江的比赛中速度能达到 2.19m/s 不现实,因此此人以和岸边垂直的方向游不能到达终点。
\\ \indent ㈡、问题二第○2 小问:为分析两次抢渡长江挑战赛到达终点人数百分比的差别,本文分别求出两次挑战赛成功到达终点速度最小值及方向和以人类极限速度渡江对应的角度进行相关的比较。为此,分别建立 1934、2002 年渡江挑战赛模型,如图四、图五所示:
\\(1)1934 年渡江挑战赛模型:
\begin{figure}[h]
    \centering
    \includegraphics[width=0.5\textwidth]{fi/屏幕截图 2025-03-01 212725.jpg}
\end{figure}
 据图可知,当挑战者的速度方向与合速度方向垂直时,所求人的渡江速度最小,模型计算如下:
\begin{eqnarray}
    v1=v3*\sin(a)=0.44
    \\c=90+a=103.41^{\circ}
\end{eqnarray}
即 1934 年 挑 战 者 渡 江 速 度 最 小 值 为0.44m/s,对应的方向:与水流的正方向夹角为103.41。
\\考虑到人游泳的极限速度
\begin{eqnarray}
    v1max=1.622
    \\b=\arcsin(\sin(a)*\frac{\sin(a)}{1.662})=15.68^{\circ}
    \\c=a+b=29.09^{\circ}
\end{eqnarray}
 经分析与计算可知:1934 年挑战者要想成功渡过长江到达终点,只要将速度保持在 0.44m/s—1.622m/s 之间,游泳的方向保持在与水流的正方向夹角为度22.09。 -103.41。 之间就可以,这比较容易实现。
\\(2)2002 年渡江挑战赛模型:
\\○1 对于 2002 年,最小速度求法与 1934 相似,当挑战者的速度方向与合速度方向垂直时,所求人的渡江速度最小,模型计算如下:
\begin{eqnarray}
    v1=v3*\sin(a)=1.43
    \\c=a+b=139.24^{\circ}
\end{eqnarray}
考虑到人游泳的极限速度
\begin{eqnarray}
    v1max=1.622
    \\b=\arcsin(\sin(a)*\frac{\sin(a)}{1.662})=61.95^{\circ}
    \\c=a+b=111.19^{\circ}
\end{eqnarray}
 经分析与计算可知:2002 年的挑战者要想成功渡过长江到达终点,需要将速度控制在 1.43m/s—1.622m/s 之间,游泳的方向保持在与水流正方向夹角为111.19。 -139.24。 之间,而这对一般挑战者来说并不容易做到。
\\问题二第○小问综合以上分析,1934 年 和 2002 年能游到终点的人数的百分比有如此大的差别,原因有以下几点:
\\(1)1934 年与 2002 相比,其起点正对面与终点的水平距离大得多,如果在游泳方向相同的情况下就为渡江赢得了更多的时间,这样,1934 年的挑战者就可以以相对很小的速度成功渡江达到终点,所以成功的比例比 2003 年大得多。
\\(2) 1934 年挑战者有可能渡江到达终点的游泳方向比 2002 年大了42.27。 ,这样使得 1934 年比 2002 年由于游泳方向不对而一开始就输掉挑战赛的比例大大降低,所以成功的比例比 2002 年大得多。
\\(3)比赛时间的差异,1934 年是 9 月,2002 年是 5 月,两者天气条件,水情,水性状况不同。
\\㈢、问题二第○小问:成功到达终点的选手的条件:
1934 年 挑 战 者 要 想 成 功 渡 过 长 江 到 达 终 点 , 只 要 将 速 度 保 持 在
0.44m/s—1.622m/s 之 间 , 游 泳 的 方 向 保 持 在 与 水 流 的 正 方 向 夹 角 为 度22.09。 -103.41。 之间就可以。
\\2002 年 的 挑 战 者 要 想 成 功 渡 过 长 江 到 达 终 点 , 需 要 将 速 度 控 制 在1.43m/s—1.622m/s 之间,游泳的方向保持在与水流正方向夹角为111.19。 -139.24。之间就可以成功到达终点。
\subsection{问题三的建模}
㈠、问题分析:因为游泳者的速度大小不变,要想渡江成功到达终点,就必须控制游泳者的方向,才有可能成功,建立模型如图六所示:
\begin{figure}[h]
    \centering
    \includegraphics[width=0.5\textwidth]{fi/屏幕截图 2025-03-01 221959.jpg}
\end{figure}
\section{模型的优缺点分析}
在实际情况中,由于各种因素的作用,如,水速对选手速度的影响,选手判断方向的误差等,选手将不能严格按照模型四中方案提供的路线前进,结果可能导致到达不了终点。
\begin{thebibliography}{100}

    \bibitem{ref1}赵静,《数学建模与数学实验》,北京:高等教育出版社,2008 年
    
   
\end{thebibliography}
     




\end{document}  