\documentclass{article}
\usepackage[utf8]{inputenc}
\usepackage{graphicx}
\usepackage{tabularx}
\usepackage{ctex}
\usepackage{gensymb}
\usepackage{amsmath}
\usepackage{booktabs} 
\usepackage{enumerate}
\usepackage{hyperref}
\title{层次分析法}
\author{李俊霖}
\date{\today}

\begin{document}

\maketitle
\section{问题提出}
周末干什么已经和中午吃什么一样,成为困扰大学生的重大疑难问题之一,因此,大学生李华最近密集的安排将他搞的晕头转向,
他想求助你帮他根据给出的条件为他安排最优的周末安排,请你根据此进行数学建模。
\\ \indent 题目信息: 目标 周末安排;准则 紧急程度,心理接受度,精力占用度;方案 写数模作业,出去玩,在宿舍睡觉
\\ \indent 判断矩阵
\[
W_{CB_1} = \begin{pmatrix}
1 & 4 & 4 \\
0.25 & 1  & 1 \\
0.25 & 1  & 1
\end{pmatrix}
\]
\[
W_{CB_2} = \begin{pmatrix}
1 & 0.2 & 0.5 \\
5& 1  & 2.5 \\
2 & 0.4  & 1
\end{pmatrix}
\]
\[
W_{CB_3} = \begin{pmatrix}
1 & 2 & 10 \\
0.5& 1  & 5 \\
0.1 & 0.3  & 1
\end{pmatrix}
\]
\[
W_{BA} = \begin{pmatrix}
1 & 2 & 4 \\
0.5& 1  & 2 \\
0.25 & 0.5  & 1
\end{pmatrix}
\]
\section{模型假设}
1. 除题目外其他信息对于模型没有其他影响
\\ \indent 2. 方案层与准则层的重要程度取决于判断矩阵
\\ \indent 3. 模型应用层次分析法建立
\section{模型建立}
为得到方案层中$C_1$,$C_2$,$C_3$对应的权重$W_{C_1A}$,$W_{C_2A}$,$W_{C_3A}$我们可以逐层地确定相邻层间的相对
权重矩阵,并通过多个权重矩阵的相乘,确定最终的权重。
\\ \indent 对于题目信息建立层次模型,如下图
\begin{figure}[h]
    \centering
    \includegraphics[width=0.5\textwidth]{fi/屏幕截图 2025-03-16 110807.jpg}
\end{figure}
建立矩阵乘法公式
\begin{equation}
    \left[
    \begin{array}{c}
    W_{C_1A} \\
    W_{C_2A} \\
    W_{C_3A}
    \end{array}
    \right]
    =
    \left[
    \begin{array}{ccc}
    W_{C_1B_1} & W_{C_1B_2} & W_{C_1B_3} \\
    W_{C_2B_1} & W_{C_2B_2} & W_{C_2B_3} \\
    W_{C_3B_1} & W_{C_3B_2} & W_{C_3B_3}
    \end{array}
    \right]
    \left[
    \begin{array}{c}
    W_{B_1A} \\ 
    W_{B_2A} \\
    W_{B_3A}
    \end{array}
    \right]
    \end{equation}
即
\begin{equation}
  W_{CA}=W_{CB}*W_{BA}
\end{equation}
由 $W_{CA}=W_{CB}*W_{BA}$ 可见所要求的C对A的权重矩阵为C对
B的权重矩阵与B对A的权重矩阵的乘积。
\\ \indent 对判断矩阵W进行特征值分解权重向量($W_{C_1B}$,$W_{C_2B}$,$W_{C_3B}$)取为
矩阵模最大特征值对应的特征向量,这样的特征向量被
称为主特征向量。由于权重向量各分量之和应该为1,
故我们将各分量分别除以各分量之和,即进行归一化。
\\ \indent 最终,我们计算得到三种候选项的权重向量
\\ \indent 最终到得到了C和A层间的权重矩阵,并进行选择
\section{模型求解}

\subsection{求解$W_{CB}$}
\begin{enumerate}[(1)]
\item 求解$W_{CB_1}$,
    原矩阵
 \[
 W_{CB_1} = \begin{pmatrix}
 1 & 4 & 4 \\
 0.25 & 1  & 1 \\
 0.25 & 1  & 1
 \end{pmatrix}
 \]
 计算过程
 \begin{figure}[h]
    \centering
    \includegraphics[width=0.5\textwidth]{fi/屏幕截图 2025-03-16 130454.jpg}
 \end{figure}
 计算得:
 \begin{equation}
    \left[
    \begin{array}{c}
    W_{C_1B_1} \\
    W_{C_2B_1} \\
    W_{C_3B_1}
    \end{array}
    \right]
    =
    \left[
    \begin{array}{c}
    0.6666\\ 
    0.1667 \\
    0.1667
    \end{array}
    \right]
    \end{equation}

\item 求解$W_{CB_2}$
    原矩阵
 \[
 W_{CB_2} = \begin{pmatrix}
 1 & 0.2 & 0.5 \\
 5& 1  & 2.5 \\
 2 & 0.4  & 1
 \end{pmatrix}
 \]
 计算过程
    \begin{figure}[h]
        \centering
        \includegraphics[width=0.5\textwidth]{fi/屏幕截图 2025-03-16 133359.jpg}
    \end{figure}
 计算得:
    \begin{equation}
        \left[
        \begin{array}{c}
        W_{C_1B_2} \\
        W_{C_2B_2} \\
        W_{C_3B_2}
        \end{array}
        \right]
        =
        \left[
        \begin{array}{c}
        0.1250\\ 
        0.6250 \\
        0.2500
        \end{array}
        \right]
        \end{equation}
    \item 求解$W_{CB_3}$
    原矩阵
    \[
    W_{CB_3} = \begin{pmatrix}
        1 & 2 & 10 \\
        0.5& 1  & 5 \\
        0.1 & 0.2  & 1
        \end{pmatrix}
    \]
    计算过程
    见附录
    \\计算得:
        \begin{equation}
            \left[
            \begin{array}{c}
            W_{C_1B_2} \\
            W_{C_2B_2} \\
            W_{C_3B_2}
            \end{array}
            \right]
            =
            \left[
            \begin{array}{c}
            0.6250\\ 
            0.3125 \\
            0.0625
            \end{array}
            \right]
        \end{equation}
   

        
\end{enumerate}

\subsection{求解$W_{BA}$}
原矩阵
    \[
     W_{BA} = \begin{pmatrix}
     1 & 2 & 4 \\
     0.5& 1  & 2 \\
     0.25 & 0.5  & 1
     \end{pmatrix}
    \]
    计算过程
    \begin{figure}[h]
        \centering
        \includegraphics[width=0.5\textwidth]{fi/屏幕截图 2025-03-16 134258.jpg}
    \end{figure}
        计算得:
        \begin{equation}
            \left[
            \begin{array}{c}
            W_{C_1B_2} \\
            W_{C_2B_2} \\
            W_{C_3B_2}
            \end{array}
            \right]
            =
            \left[
            \begin{array}{c}
            0.5715\\ 
            0.2857 \\
            0.1428
            \end{array}
            \right]
            \end{equation}


\subsection{矩阵相乘} 
\begin{equation}
    \left[
    \begin{array}{c}
    W_{C_1A} \\
    W_{C_2A} \\
    W_{C_3A}
    \end{array}
    \right]
    =
    \left[
    \begin{array}{ccc}
        0.6666 & 0.1250 & 0.6250\\ 
        0.1667 & 0.6250 & 0.3125 \\
        0.1667 & 0.2500 & 0.0625
    \end{array}
    \right]
    \left[
    \begin{array}{c}
        0.5715\\ 
        0.2857 \\
        0.1428
    \end{array}
    \right]
    \end{equation}
得
    \begin{equation}
        \left[
        \begin{array}{c}
        W_{C_1A} \\
        W_{C_2A} \\
        W_{C_3A}
        \end{array}
        \right]
        =
        \left[
        \begin{array}{c}
            0.5059\\ 
            0.3185 \\
            0.1756
        \end{array}
        \right]
        \end{equation}
\section{选择结果}
所以得到最终结果为写数学建模作业,因此,我们得出李华周末应该写数学建模作业
\section{附录}
$M_{CB_3}$计算过程
     \begin{figure}[hb]
        \centering
        \includegraphics[width=0.5\textwidth]{fi/屏幕截图 2025-03-16 133909.jpg}
     \end{figure}
\end{document}  

